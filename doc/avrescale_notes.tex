%\documentclass[12pt]{article}
\documentclass[reprint]{revtex4-1}
\usepackage{amsmath}
\usepackage{mathtools}
\usepackage{upgreek}
\usepackage[table,usenames,dvipsnames]{xcolor}
\usepackage{hyperref}

\hypersetup{
  colorlinks,
  linkcolor={red!30!black},
  citecolor={green!20!black},
  urlcolor={blue!80!black}
}


\definecolor{DarkBlue}{RGB}{0,0,64}
\definecolor{DarkBrown}{RGB}{64,20,10}
\definecolor{DarkGreen}{RGB}{0,64,0}
\definecolor{DarkPurple}{RGB}{64,0,42}
\definecolor{LightGray}{gray}{0.85}
% annotation macros
\newcommand{\red}[1]{{\color{red} #1}}
\newcommand{\repl}[2]{{\color{gray} [#1] }{\color{blue} #2}}
\newcommand{\add}[1]{{\color{blue} #1}}
\newcommand{\del}[1]{{\color{gray} [#1]}}
\newcommand{\note}[1]{{\color{DarkGreen}\footnotesize \textsc{Note.} #1}}
\newcommand{\answer}[1]{{\color{DarkBlue}\footnotesize \textsc{Answer.} #1}}
\newcommand{\summary}[1]{{\color{DarkPurple}\footnotesize \textsc{Summary.} #1}}



\begin{document}



\title{Notes on ``Adaptive velocity scaling for an asymptotic
microcanonical ensemble''}
\author{}
%\date{\vspace{-7ex}}

\maketitle



\section{\label{sec:model}Ratio $\gamma$ in a model system}



The ratio defined in Eq. \eqref{eq:gamma_def}
depends on the potential energy.
%
Consider the model Hamiltonian of $\mathbf x = (\mathbf r, \mathbf v)$,
\begin{equation}
  H(\mathbf x)
  =
  \frac{\mathbf v^2} { 2 }
  +
  \left( \frac{\mathbf r^2} { 2 } \right)^\theta
  ,
\end{equation}
%
where $K = \frac 1 2 {\mathbf v}^2$ and
$U = \frac 1 2 {\mathbf r}^2$
are the kinetic and potential energy, respectively,
and $\theta$ is a positive free parameter.
Then
\begin{align*}
  \Omega(E)
  &=
  C^2
  \int
    \delta\left( K + U^\theta - E \right) \,
    K^{\frac{ N_f } 2 - 1} \, dK \, U^{\frac{ N_f } 2 - 1} \, dU
  \\
  &=
  \frac{ C^2 } { \theta }
  \int
  K^{\frac{ N_f } 2 - 1} \, (E - K)^{\frac{ N_f }{ 2 \, \theta } - 1}
    \, dK
  \\
  &=
  \frac{ C^2 }{ \theta } \,
  B\left( \frac{ N_f } {2 \, \theta}, \frac{ N_f } 2 \right)
  E^{ \frac{ N_f }{2 \, \theta} + \frac{N_f}{2} - 1 }
  ,
\end{align*}
where
%
$C = 2 \, \pi^{N_f/2} / \Gamma\left( N_f / 2 \right)$,
and
$B(a, b) = \Gamma(a) \, \Gamma(b) / \Gamma(a+b)$
is the beta function.
%
Then, we have
\begin{align*}
\beta(E)
&=
\left(
  \frac{ N_f } { 2 \, \theta } + \frac{ N_f } 2 - 1
\right)
E^{-1}
,
\\
\beta'(E)
&=
-
\left(
  \frac{ N_f } { 2 \, \theta } + \frac{ N_f } 2 - 1
\right)
E^{-2}
\\
\left\langle
  \frac{
    N_f - 2
  }
  {
    2 \, K^2
  }
\right\rangle
&=
  E^{-2}
\left.
  B\left( \frac{ N_f } { 2  \, \theta } - 2, \frac{ N_f } { 2 } \right)
\middle/
  B\left( \frac{ N_f } { 2  \, \theta }, \frac{ N_f } { 2 } \right)
\right.
\\
&=
\frac{ \frac{ N_f } 2 + \frac{ N_f }{2 \, \theta} - 1 }
     { E^2 }
\frac{ \frac{ N_f } 2 + \frac{ N_f }{2 \, \theta} - 2 }
     { \frac{ N_f } 2 - 2 }
.
\end{align*}
This means the ratio defined in Eq. \eqref{eq:gamma_def}
$$
\gamma
=
\frac
{
  \frac{ N_f } 2 - 2
}
{
  \frac{ N_f } 2 + \frac{N_f}{2 \, \theta} - 2
}
,
$$
which lies between $0$ and $1$.



%\bibliographystyle{abbrv}
\bibliography{simul}
\end{document}
